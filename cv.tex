%!TEX TS-program = xelatex
\documentclass[]{friggeri-cv}
%\addbibresource{bibliography.bib}

\begin{document}

\ifenglish
\header{Beatriz}{Hernández}
       {Data Scientist}


% In the aside, each new line forces a line break
\begin{aside}
  \section{contact}
    28002 Madrid
    Spain
    ~
    \href{mailto:---@gmail.com}{---@gmail.com}
    \href{skype:<--->[add]}{skype://---}
    \href{tel:----}{+----}
  \section{languages}
    spanish
    english
  \section{programming languages}
    Python (SciProgramming), R, Haskell
    MySQL, NoSQL, VBA
    BASH
    {\color{red} $\varheartsuit$} Tex
    {\color{red} $\varheartsuit$} D3js
    HTML, CSS, JavaScript, NodeJS
  \section{tools}
    JIRA, Trello
    Git
\end{aside}

\section{interests}

modelling, statistical analysis, data visualization, graph theory

\section{experience}

\begin{entrylist}
  \entry
    {2016 - 2018}
    {DatMean, Madrid}
    {Data Scientist}
    {Leader of the Data Science team in DatMean. Supervising of DS projects which include reporting, analysis, machine learning and deep learning algorithmics. Development of tools based on data with application for a marketing trading desk and a data marketplace. Development of tools for the trader to optimize time supervising campaings. R\&D in online behaviour targeted on increase a website audience.}
  \entry
    {2016}
    {isnotTV, remote}
    {Data Scientist}
    {Development of a bot, which answers cinema related questions. Managing and development of a movie recommender based. Prepare graphics for cinema blog articles. Techonologies: MongoDB, R, Python, and ElasticSearch.}
  \entry
    {2015 - 2016}
    {Havas Media, Madrid}
    {Data Scientist}
    {Data mining and data visualization for clients Gas Natural and Decathlon among others. Development of planning tools for media planners. Technologies: R, Python, Tableau, VBA for Excel, and D3js.}
\end{entrylist}

\section{organisation}

\begin{entrylist}
  \entry
    {2016 - 2017}
    {R-Ladies Madrid}
    {Founder \& co-organiser}
    {R-Ladies Madrid is a local chapter of R-Ladies Global. R-Ladies Madrid exists to promote gender diversity in the R community, both in Spain and worldwide. R-Ladies Global is funded by the R Consortium-Linux Foundation.}
  \entry
    {2016-06-03}
    {Call of Data}
    {Organisation}
    {First datathon organised by R-Ladies Madrid. Our main goal is to promote gender diversity in technical competitions. A one track only filled with talks in the morning and the competition in the afternoon.}
\end{entrylist}

\section{talks}

\begin{entrylist}
  \entry
    {2017-11-11}
    {Analytics with R, the tidyverse point of view}
    {GDG DevFest Lisbon}
    {You are interested in statistics, in data, in analytics, but heard that R is not a friendly language. That changed over the last years with the tidyverse universe, a fast, friendly, and high level way of doing stats in R. From importing your data, to functional programming, tidyverse will help you to start and work with R.}
  \entry
    {2017-11-06}
    {WikiNews, a collaboration story}
    {GitHub Constellation}
    {How three tech communities gave life to the WikiNews project. We brought women together to collaborate on a project that joined R coding, Node know-how, and website building skills.}
  \entry
    {2017-05-13}
    {JavaScript for a Data Scientist}
    {JSDayES}
    {Difficulties and steps for a Data Scientist to use JavaScript in his/her day to day.}
  \entry
    {2017-04-13}
    {Math in Escher paintings.}
    {CARTO}
    {Historical context and introduction to the math used in Escher paintings.}
  \entry
    {2017-02-21}
    {Hands on with LSTM Networks}
    {Data Science Madrid}
    {Intro to time series prediction with Python and Deep Learning technology. \href{https://github.com/chucheria/20170221_DSM-Workbook}{Resources}.}
\end{entrylist}

\section{education}

\begin{entrylist}
  \entry
    {2015}
    {BSc in Mathematics}
    {Complutense University of Madrid}
    {Graduate thesis: Topological graph theory: from Königsberg to urban traffic.}
\end{entrylist}

%\section{publications}


% \printbibsection{article}{article in peer-reviewed journal}
% \begin{refsection}
%   \nocite{*}
%   \printbibliography[sorting=chronological, type=inproceedings, title={international peer-reviewed conferences/proceedings}, notkeyword={france}, heading=subbibliography]
% \end{refsection}
% \begin{refsection}
%   \nocite{*}
%   \printbibliography[sorting=chronological, type=inproceedings, title={local peer-reviewed conferences/proceedings}, keyword={france}, heading=subbibliography]
% \end{refsection}
% \printbibsection{misc}{other publications}
% \printbibsection{report}{research reports}

\else

\header{Beatriz}{Hernández}
       {Data Scientist}


% In the aside, each new line forces a line break
\begin{aside}
  \section{contacto}
    28002 Madrid
    España
    ~
    \href{mailto:---@gmail.com}{---@gmail.com}
    \href{skype:<--->[add]}{skype://---}
    \href{tel:----}{+----}
  \section{idiomas}
    español nativo
    inglés fluido
    alemán principiante
  \section{programación}
    Python (SciProgramming), R, Haskell
    MySQL, NoSQL, VBA
    BASH
    {\color{red} $\varheartsuit$} Tex
    {\color{red} $\varheartsuit$} D3js
    HTML, CSS, JavaScript, NodeJS
  \section{otras herramientas}
    JIRA, Trello
    Git
\end{aside}

\section{intereses}

modelos, análisis estadístico, visualización de datos, teoría de grafos

\section{experiencia}

\begin{entrylist}
  \entry
    {2016 - 2018}
    {DatMean, Madrid}
    {Data Scientist}
    {Liderazgo de proyectos de Data Science: reportes, análisis, machine learning, deep learning. Organización del equipo de Data Science de DatMean. Desarrollo de aplicaciones orientadas al trading desk de campañas online basadas en datos. Creación de herramientas para ayudar al trader en la optización de las campañas. I+D en comportamiento y navegación online enfocado a la ampliación de audiencia de un sitio web.}
  \entry
    {2016}
    {isnotTV, remoto}
    {Data Scientist}
    {Desarrollo de un bot que da respuesta a preguntas sobre cine. Desarollo y mantenimiento de un recomendador de películas basado en machine learning. Preparado de gráficos para artículos de un blog de cine. Tecnologías: MongoDB, R, Python y ElasticSearch.}
  \entry
    {2015 - 2016}
    {Havas Media, Madrid}
    {Data Scientist}
    {Minería y visualización de datos para clientes Gas Natural y Decathlon entre otros. Desarrollo de herramientas de planifación de publicidad digital para media planners. Tecnologías: R, Python, Tableau, VBA y D3js.}
\end{entrylist}

\section{organización}

\begin{entrylist}
  \entry
    {2016 - 2017}
    {R-Ladies Madrid}
    {Fundadora y co-organizadora}
    {R-Ladies Madrid es una rama local parte de R-Ladies Global, una comunidad open source donde mujeres nos apoyamos y nos ayudamos a crecer dentro de la comunidad de R. R-Ladies Global es un proyecto financiado por el R Consortium-Linux Foundation.}
  \entry
    {2016-06-03}
    {Call of Data}
    {Organización}
    {Primer datathon organizado por R-Ladies Madrid en el que se fomenta la diversidad de género en las competiciones técnicas. Un evento de un solo día, cargado de ponencias por la mañana y la competición por la tarde.}
\end{entrylist}

\section{ponencias}

\begin{entrylist}
  \entry
    {2017-11-11}
    {Analytics with R, the tidyverse point of view. English.}
    {GDG DevFest Lisbon}
    {You are interested in statistics, in data, in analytics, but heard that R is not a friendly language. That changed over the last years with the tidyverse universe, a fast, friendly, and high level way of doing stats in R. From importing your data, to functional programming, tidyverse will help you to start and work with R.}
  \entry
    {2017-11-06}
    {WikiNews, a collaboration story. English.}
    {GitHub Constellation}
    {How three tech communities gave life to the WikiNews project. We brought women together to collaborate on a project that joined R coding, Node know-how, and website building skills.}
  \entry
    {2017-05-13}
    {JavaScript para un Data Scientist}
    {JSDayES}
    {Exposición de las dificultades de un data scientist a la hora de usar JavaScript y posibles próximos pasos en este campo.}
  \entry
    {2017-04-13}
    {Las matemáticas de Escher}
    {CARTO}
    {Introducción a las matemáticas de los cuadros de Escher y el contexto histórico.}
  \entry
    {2017-02-21}
    {Hands on with LSTM Networks}
    {Data Science Madrid}
    {Introducción a las series temporales en Python con tecnologías de Deep Learning. \href{https://github.com/chucheria/20170221_DSM-Workbook}{Recursos}.}
\end{entrylist}

\section{educación}

\begin{entrylist}
  \entry
    {2015}
    {Grado en matemáticas}
    {Universidad Complutense de Madrid}
    {TFG: Teoría de grafos: desde Königsberg al tráfico urbano.}
\end{entrylist}

% \section{cursos}

% \begin{entrylist}
%   \entry
%       {2016}
%       {MOOC: Data Science Specialization}
%       {Coursera / John Hopkins University}
%       {10 cursos sobre distintos temas dentro de la especialización de Data Science.}
%   \entry
%     {2015}
%     {Movilidad y transporte sostenible en Smart Cities}
%     {Universidad Complutense}
%     {Información sobre las tendencias más innovadoras acerca de transporte y movilidad sostenibles}
% \end{entrylist}

%\section{publicaciones}


% \printbibsection{article}{article in peer-reviewed journal}
% \begin{refsection}
%   \nocite{*}
%   \printbibliography[sorting=chronological, type=inproceedings, title={international peer-reviewed conferences/proceedings}, notkeyword={france}, heading=subbibliography]
% \end{refsection}
% \begin{refsection}
%   \nocite{*}
%   \printbibliography[sorting=chronological, type=inproceedings, title={local peer-reviewed conferences/proceedings}, keyword={france}, heading=subbibliography]
% \end{refsection}
% \printbibsection{misc}{other publications}
% \printbibsection{report}{research reports}


\fi

\end{document}
