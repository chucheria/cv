%!TEX TS-program = xelatex
\documentclass[]{friggeri-cv}
\addbibresource{bibliography.bib}


\begin{document}

\ifenglish
\header{Beatriz}{Hernández}
       {}


% In the aside, each new line forces a line break
\begin{aside}
  \section{about}
    28002 Madrid
    Spain
    ~
    \href{mailto:beaherlorca@gmail.com}{beaherlorca@gmail.com}
    \href{https://github.com/chucheria}{github://chucheria}
  \section{languages}
    fluid english
    german notions
  \section{programming}
    {\color{red} $\varheartsuit$} Tex
    Matlab \& Mathematica
    Python (numpy, pandas), R
    MySQL, VBA
\end{aside}

\section{interests}

\section{education}

\begin{entrylist}
  \entry
    {2015}
    {B.S. in Mathematics}
    {Complutense University of Madrid}
    {Graduate thesis: \emph{Topological graph theory: from Königsberg to urban traffic.}}
  \entry
    {2010 - 2011}
    {B.S. in Mathematics}
    {Graz University of Technology}
    {erasmus scholarship}
\end{entrylist}

\section{experience}

\begin{entrylist}
  \entry
    {2015}
    {CloudMas, Madrid}
    {Data Science Internship}
    {Developing of dynamic reports with Javascript data visualization library D3. Web analytics. Machine learning with Python library OpenCV.}
  \entry
    {2013-2014}
    {Vodafone, Madrid}
    {Continual Service Improvement Internship}
    {KPIs analysis and monitoring, projects organization and follow up meetings, data analysis, management of data bases.}
\end{entrylist}

\section{applications}

\begin{entrylist}
  \entry
    {2014}
    {Excel in depth}
    {}
    {Microsoft Excel in depth}
  \entry
    {2014}
    {Access in depth}
    {}
    {Microsoft Access in depth}
  \entry
    {2015}
    {The Analytics Edge}
    {}
    {Analytics methods in R}
\end{entrylist}

\section{publications}


% \printbibsection{article}{article in peer-reviewed journal}
% \begin{refsection}
%   \nocite{*}
%   \printbibliography[sorting=chronological, type=inproceedings, title={international peer-reviewed conferences/proceedings}, notkeyword={france}, heading=subbibliography]
% \end{refsection}
% \begin{refsection}
%   \nocite{*}
%   \printbibliography[sorting=chronological, type=inproceedings, title={local peer-reviewed conferences/proceedings}, keyword={france}, heading=subbibliography]
% \end{refsection}
% \printbibsection{misc}{other publications}
% \printbibsection{report}{research reports}

\else 

\header{Beatriz}{Hernández}
       {}


% In the aside, each new line forces a line break
\begin{aside}
  \section{info}
    28002 Madrid
    España
    ~
    \href{mailto:beaherlorca@gmail.com}{beaherlorca@gmail.com}
    \href{https://github.com/chucheria}{github://chucheria}
  \section{idiomas}
    inglés fluido
    alemán medio
  \section{programación}
    {\color{red} $\varheartsuit$} Tex
    Matlab \& Mathematica
    Python (numpy, pandas), R
    MySQL, VBA
\end{aside}

\section{intereses}

\section{educación}

\begin{entrylist}
  \entry
    {2015}
    {Grado {\normalfont en matemáticas}}
    {Universidad Complutense de Madrid}
    {TFG: \emph{Teoría de grafos: desde Königsberg al tráfico urbano}}
  \entry
    {2010-2011}
    {Grado {\normalfont en matemáticas}}
    {Universidad Técnica de Graz}
    {beca erasmus}
\end{entrylist}

\section{experiencia}

\begin{entrylist}
  \entry
    {04–07 2015}
    {CloudMas, Madrid}
    {Beca en análisis de datos}
    {\emph{Representación de datos mediante D3. Análisis de datos web. Proyectos en aprendizaje automático de máquinas.}}
  \entry
    {2013-2014}
    {Vodafone, Madrid}
    {Beca en mejora continua}
    {\emph{Auditoría interna, seguimiento y análisis de KPIs, coordinación y participación en proyectos de mejora de servicios, análisis de datos, administración y creación de bases de datos.}}
\end{entrylist}

\section{cursos}

\begin{entrylist}
  \entry
    {2014}
    {Excel in depth}
    {}
    {{Manejo de la herramienta Excel en profundidad incluyendo fórmulas, fomato, tablas dinámicas y macros.}
  \entry
    {2014}
    {Access in depth}
    {}
    {Manejo de la herramienta Access en profundidad incluyendo queries, formularios y reportes.}
  \entry
    {2015}
    {The Analytics Edge}
    {}
    {Análisis de datos en R}
\end{entrylist}

\section{publications}


% \printbibsection{article}{article in peer-reviewed journal}
% \begin{refsection}
%   \nocite{*}
%   \printbibliography[sorting=chronological, type=inproceedings, title={international peer-reviewed conferences/proceedings}, notkeyword={france}, heading=subbibliography]
% \end{refsection}
% \begin{refsection}
%   \nocite{*}
%   \printbibliography[sorting=chronological, type=inproceedings, title={local peer-reviewed conferences/proceedings}, keyword={france}, heading=subbibliography]
% \end{refsection}
% \printbibsection{misc}{other publications}
% \printbibsection{report}{research reports}


\fi

\end{document}
